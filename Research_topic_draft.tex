%% LyX 2.1.0 created this file.  For more info, see http://www.lyx.org/.
%% Do not edit unless you really know what you are doing.
\documentclass[american,english]{article}
\usepackage[T1]{fontenc}
\usepackage{array}
\usepackage{longtable}
\usepackage{float}
\usepackage{booktabs}
\usepackage{graphicx}
\usepackage[numbers]{natbib}
\usepackage[dot]{bibtopic}

\makeatletter

%%%%%%%%%%%%%%%%%%%%%%%%%%%%%% LyX specific LaTeX commands.
%% Because html converters don't know tabularnewline
\providecommand{\tabularnewline}{\\}

%%%%%%%%%%%%%%%%%%%%%%%%%%%%%% User specified LaTeX commands.
\usepackage{url}

\@ifundefined{showcaptionsetup}{}{%
 \PassOptionsToPackage{caption=false}{subfig}}
\usepackage{subfig}
\makeatother

\usepackage{babel}
\begin{document}

\title{General understanding towards persuasive elements in phishing email
- Research Topic}

\maketitle
\begin{center}
\textbf{Nurul Akbar}
\par\end{center}

\begin{center}
\textbf{s1192523}
\par\end{center}

\begin{center}
\textbf{University of Twente}
\par\end{center}


\part*{Motivation and background}

With the enormous growing trends of information technology in modern
generation, the evolution of digital era has become more mature in
the sense of effectiveness and easiness for societies. Trusted entities
such as financial institutions may offer their products and services
to the public through the Internet. Consequently, human society has
been enthusiastically eager to utilize technology means such as emails,
websites, online payment system, social networks to achieve their
tasks efficiently, affordable and more relevant. However, the advancement
in information and communication technology has been a double-edged
sword. It becomes easier to get personal information about someone
in the cyber world. Cyber criminals see this opportunity as a way
to manipulate consumers and exploit their confidential information
such as usernames, passwords, bank account information, credit card
or social security numbers

One of the well known cyber crimes is called phishing. Phishing attacks
aim to gain a financial benefit by masquerading legitimate institutions
\citep{jakobsson:2006}. Moreover, phishing techniques may associated
with fake emails and fake websites, however, the essence of this malicious
behavior is the art of social engineering and deception \citep{jakobsson:2006}\citep{blythe2011f}\citep{dhamija2006phishing}\citep{james:2005}\citep{jagatic2007social}
i.e. how to trick the potential victims into disclosing their sensitive
information. To illustrate one specific example; a phisher impersonates
a well-known bank institutions and sends fake email announcing there
is system upgrade to its customers, so the phisher would ask the customers
to verify their usernames, passwords and credit card numbers in order
to complete the system upgrade. In worst case, a clueless bank customer
would have no idea and fail to ascertain that legitimate bank institutions
would never ask for such things. Consequently, an unsuspecting victim
might took the bait into divulging their sensitive information to
the perpetrators. More concrete mechanism to carry out phishing attacks
is that when the phishers send a large number of phishing emails which
include malicious link or URL which redirect recipient to a fake website.
A phishing email might be resembling a legitimate email so that unsuspecting
victim might think that it is a genuine. By looking like a genuine
email, it would help to circumvent the filtering system \citep{ma2009detecting}.
One of phishing email methods is providing a malicious URL within
its content, however, filtering systems of phishing email may have
access to blacklist database of URLs that are currently being pushed,
such as crowd sourced Phishtank. So the phishers might diverge the
URL to bypass the matches.

To make phishing email efficient, its context might require potential
victim to urgently act upon it, for example an email informs about
suspended account in a banking website. Phishing attacks are carried
out using fake emails, fake websites and often phishers may exploit
the end user\textquoteright s web browser to bypass warnings and URL
information. Several countermeasures have been studied to detect phishing
emails, such as the technique of looking for bit string that are previously
determined as spams and phishing emails \citep{wardman:2009}. The
common characteristics of fake email sent by the phishers would be
misleading hyperlinks and misleading header information \citep{zhang:2006,zhang:2007}.
Furthermore, one of the non technical approach to defense against
cyber fraud such as security awareness education to the public to
ignore links within an email, even though the source of the email
appear to be legitimate \citep{sheng:2010}. 

The success of phishing attack through distributed emails is determined
by the response of the unsuspecting recipients. User decisions to
click a link or open an attachment in an email might be influenced
by how strong a phisher can persuade a victim. However, current countermeasures
in a phishing email so far do not consider psychoanalytic approach
such as persuasion techniques, which it might be important for user
decision. Thus, characterization and the relevancy of phishing email
properties as well as the relevancy of persuasion techniques with
these structural properties could fill the void.

We begin our research topic by understanding what is phishing itself;
how phishing emerged in term of history and conduct literature surveys
on phishing definition.\foreignlanguage{american}{ Secondly, we will
explore its damage in term of money. Thirdly, we will conduct an overview
of its modus operandi. In the fourth section, we will introduce a
brief explanation on types of phishing. In the fifth section, an understanding
of bad neighborhoods on phishing will be discussed. In the sixth section
we will address the general phishing countermeasures. Lastly, in the
last section of this chapter, human factor in phishing attacks will
be discussed.}


\section*{What is phishing?}

\selectlanguage{american}%
While the Internet has brought convenience to many people for exchanging
information, it also provides opportunities to carry malicious behavior
such as online fraud on massive scale with a little cost to the attackers.
The attackers can manipulate the Internet users instead of computer
system (hardware or software) that significantly increase the barriers
of technological crime impact. Such human centered attacks could be
done by social engineering. According to Jakobsson, et al. phishing
is a form of social engineering that aim{\small{}s} to retrieve credential
from online users by mimicking trustworthy and legitimate institutions
\citep{jakobsson:2006}. Phishing has a similar basic principle as
\textquoteleft fishing\textquoteright{} in the physical world. Instead
of fish, online users are lured by authentic looking communication
and hooked by authentic looking websites. Not only that, online users
also may be lured by respond to a phishing email, either replying
or clicking an obfuscated link within its content. There are diverse
definitions of phishing in our literature reviews, therefore, we would
like to discuss about its universal definition in later section. However,
one of phishing definitions according to Oxford dictionary:
\begin{quote}
``A fraudulent practice of sending emails purporting to be from reputable
companies in order to induce individuals to reveal personal information,
such as passwords and credit card numbers, online'' \citep{oxford}. 
\end{quote}
Several studies suggest that phishing is form of online criminal activity
by using social engineering techniques \citep{jakobsson:2006}\citep{workman:2008}\citep{jagatic2007social}\citep{chandrasekaran:2006}.
An individual or a group who uses this technique is called \emph{Phisher(s}).
After successfully gaining a sensitive information from the victim,
phishers use this information to access victim\textquoteright s financial
accounts or committing credit card frauds. However, to formalize the
damage of phishing in term of money is a challenging task. We will
briefly explore the cost of phishing attacks in the later section. 

Furthermore, the technique or modus operandi of phishing may vary,
but the most common technique of phishing attacks done by using fraudulent
emails and websites\citep{james:2005}. A fraudulent website is designed
in such a way that it may be identical to its legitimate target. While
it may be true, phishing website also could be completely different
with its target as there is no level of identicalness. With this in
mind, preliminary analysis on what changed or added in phishing website
would be conducted in the later section. In the following subsections,
we will introduce how phishing was originally came about and how current
literatures formally define phishing.


\subsection*{The History}

The first time the term \textquotedbl{}phishing\textquotedbl{} was
published by the AOL UseNet Newsgroup on January 2, 1996 and was started
to expand in 2004 \citep{phishorg}. Since then, we considered phishing
development in cyberspace has been flourishing by phishers to make
profit. Total losses due to phishing in 2004 reached more than U.S.
\$ 2 billion, it was involving more than 15,000 sites that become
victims \citep{fellman:2004}. We will try to discuss about direct
and indirect cost at present days in the later section. Evidently,
Jakobsson, et al. \citep{jakobsson:2006} mentioned that in the early
years of 90\textquoteright s (according to \citep{phishorg} it was
around 1995) many hackers would create bogus AOL user accounts with
automatically generated fraudulent credit card information. Their
intention to give this fake credit card information was to simply
pass the validity tests performed by AOL. By the time the tests were
passed, AOL was thinking that these accounts were legitimate and resulted
to activate them. Consequently, these hackers could freely access
AOL resources until AOL tried to actually bill the credit card. AOL
realized that these accounts were using invalid billing information,
thus deactivated the account. 

While creating false AOL user accounts with fake credit card information
was not exactly phishing attacks, but AOL\textquoteright s effort
to counter against the attacks was leading to development of phishing.
This countermeasure includes directly verifying the legitimacy of
credit card information and the associated billing identity, forced
hackers to pursue alternative way \citep{jakobsson:2006}. Hackers
were masquerading as AOL\textquoteright s employees asking to other
users for credit card information through AOL instant messenger and
email system \citep{phishorg}. Jakobsson et al. suggest that phishing
attacks were originating from this incident \citep{jakobsson:2006}.
Since such attack has not been done before, many of users have been
victimized by then. Eventually, AOL enforced warning system to the
most of its customers to be vigilant when it comes to sensitive information
\citep{phishorg}. At the present day, phishing attacks might not
only being motivated by financial gain but also political reason,
and they have been emerging not only aim to AOL users, but also any
online users. Consequently, large number of legitimate institutions
such as PayPal and eBay are being spoofed.


\subsection*{Universal definition}

Before we begin to understand deeper about how and why phishing attack
works, we will briefly explore common phishing definition. Currently,
there is no consensus definition, since almost in every research papers,
academic textbook or journals has its own definition of phishing \citep{jakobsson:2006,james:2005,tally:2004,clayton:2005,parno:2006,jakobsson:2005,dhamija2006phishing}.
Phishing is also constantly evolving, so it might be very challenging
to define its universal terminology. There is not so much study that
specifically addresses the standard of phishing definition. However,
one research conducted by Lastdrager \citep{lastdrager:2014} addressed
to achieve consensual definition of phishing. Before we comply with
one consensual phishing terminology, we will take a look at various
phishing definitions from other sources:
\begin{quote}
\emph{``Phishing is the act of sending a forged e-mail (using a bulk
mailer) to a recipient, falsely mimicking a legitimate establishment
in an attempt to scam the recipient into divulging private information
such as credit card numbers or bank account passwords''} \citep{james:2005}

\emph{``Phishing is a form of Internet scam in which the attackers
try to trick consumers into divulging sensitive personal information.
The techniques usually involve fraudulent E-mail and web sites that
impersonate both legitimate E-mail and web sites''} \citep{tally:2004}

\emph{``Phishing is an attack in which victims are lured by official
looking email to a fraudulent website that appears to be that of a
legitimate service provider''} \citep{clayton:2005}

\emph{``In phishing, an automated form of social engineering, criminals
use the internet to fraudulently extract sensitive information from
businesses and individuals, often by impersonating legitimate web
sites''} \citep{parno:2006}
\end{quote}
It is noteworthy that the definition described by James, et al, Tally,
et al, and Clayton, et al. \citep{james:2005,tally:2004,clayton:2005}
specifies that the phishers only use email as a communication channel
to trick potential victims. While it might be true because using email
would be cost effective method, but we believe that phishing is not
only characterized by one particular technological mean, as phishers
can also use any other electronic communication to trick potential
victims (i.e private message on online social network). This definition
is also similar to dictionary libraries \citep{oxford,collins,merriam}
that mention email as a medium communication between phishers and
users.

We believe that standard definition of phishing should be applicable
in most of phishing concept that are presently defined. Consequently,
the high level of abstraction and is required to build common definition
on phishing. We have convinced that the formal definition of phishing
should not focus on the technology that is being used but rather on
the technique how the deception is being conducted, the method of
an ``act'' if you will. Therefore, We follow the definition of phishing
by Lastdrager \citep{lastdrager:2014} which stated: 
\begin{quote}
\textit{``Phishing is a scalable act of deception whereby impersonation
is used to obtain information from a target''}
\end{quote}
According to Lastdrager \citep{lastdrager:2014}, to achieve this
universal definition, a systematic review of literature up to August
2013 was conducted along with manual peer review, which resulted in
113 distinct definitions to be analyzed. We thereby agree with Lastdrager
\citep{lastdrager:2014} that this definition addresses all the essential
elements of phishing and we will adopt it as universally accepted
terminology throughout our research.


\section*{\label{sec:The-cost-of}The cost of phishing attacks}

It is a challenging task to find a real cost from phishing attacks
in term of money or direct cost. This due to financial damage for
bank is only known by banks and most institutions do not share this
information with the public. Evidently, Jakobsson et al. argue that
phishing economy is consistent with black market economy and does
not advertise its successes \citep{jakobsson:2006}. On this section,
a brief explanation of direct and indirect cost on phishing attack
will be illustrated based on literature review.

According to Jakobsson et al., direct cost is depicted by the value
of money or goods that are directly stolen through phishing attack
\citep{jakobsson:2006}. While indirect cost is the costs that do
not represent the money nor goods that are actually stolen, but it
is the costs has to be paid by the people who handle these attacks
\citep{jakobsson:2006} (i.e. time, money and resources spent to reset
people password).

As we mentioned earlier, the difficulty of assessing the damage on
phishing attacks is caused by banks and institutions that keep this
information themselves and the unwillingness of many users to share
to acknowledge that they have been victimized by phishing attacks.
This happens because of fear of humiliation, financial loses, or legal
liability \citep{jakobsson:2006}. Evidently, studies estimate the
damage ranging from \$61 million \citep{herley:2009} to \$3 billion
per year \citep{mccall2007gartner} of direct losses to victims in
the US only \citep{hong:2012}\citep{moura2009scalable}. In addition,
the Gartner Group claimed to estimate of \$1.2 billion direct losses
of phishing attack to US banks and credit card companies for the year
2004 \citep{litan2004phishing}. By the 2007, it escalated to more
than \$3 billion loss \citep{mccall:2007}. The estimation also performed
by TRUSTe and Ponemon Institute that stated the cost of phishing attack
was up to \$500 millions losses in the US for the same year %
\footnote{http://www.theregister.co.uk/2004/09/29/phishing\_survey/%
}. 

However, the lack of information on how to find these numbers such
as the detailed of documentation survey conducted by Gartner group
or Ponemon makes these estimations are biased than is generally realized.
It is interesting to investigate why their estimations are repeated
without really analyzed when they appeared biased at best. One thing
come up in our mind that they might have a hidden agenda to make societies
think that the cost of phishing is high. Consequently, people would
be obligated to implement anti phishing system or engage in phishing
awareness in their company which require money. With this in mind,
we would like to emphasize that our findings on the costs of phishing
attacks are only estimations without scrutiny from academic researchers
and it might be an exaggeration. Although even if the cost of phishing
attack is zero, we believe that phishing is still a major problem
in terms of trust among users and the misuse of email as a means of
communication.

Apart from the cost of phishing attacks, we might ask how phishing
attack is carried out? is there any stages involved? The next chapter
we will review its modus operandi in term of phishing stages or phases.


\section*{Modus operandi}

As we mentioned earlier, Phishing attack is a subset of identity theft.
The modus operandi might be carried out firstly by creating a fake
website that spoofs legitimate website such as financial website,
either identical or not identical as long as the phishers get responds
from unsuspected victims. After that, the phishers will try to trick
the potential victim to submit important information such as usernames,
passwords, PINs, etc. through a fake website that they have created
or through email reply from victims. With the information obtained,
they will try to steal money from their victims if target institution
is a bank. Phishers employ variety of techniques to trick potential
victims to access their fraudulent website. One of the typical ways
is by sending illicit email in a large scale claiming to be from legitimate
institution. In the email content, they usually imitate an official-looking
logo, using good business language style and often also forge the
email headers to make it look like originating from legitimate institution.
For example, the content of the email is to inform the user that the
bank is changing its IT infrastructure, and request urgently that
the customer should update their data with the consequence of loosing
their money if the action does not take place. When the user click
the link that was on the email message, they will be redirected to
a fraudulent website, which will prompt the victim to fill in the
details of their information. While there are various techniques of
phishing attack, we will address the common phases of phishing that
we analyzed by literature survey by several studies and also we will
address our own phishing phases. These phases are compiled in Table
\ref{tab:compilation-phases}.

\begin{flushleft}
\begin{longtable}{>{\raggedright}p{12cm}}
\caption{\label{tab:compilation-phases}Compilation of phishing phases}
\tabularnewline
\toprule 
J. Hong \citep{hong:2012}\tabularnewline
\midrule
{\scriptsize{}1. Potential victims receive a phish}{\scriptsize \par}

{\scriptsize{}2. The victim may take a suggested action in the message}{\scriptsize \par}

{\scriptsize{}3. The phisher monetizes the stolen information}\tabularnewline
\midrule 
Frauenstein, et al. \citep{frauenstein:2013}\tabularnewline
\midrule 
{\scriptsize{}1. Planning}{\scriptsize \par}

{\scriptsize{}2. Email Design}{\scriptsize \par}

{\scriptsize{}3. Fabricated story}{\scriptsize \par}

{\scriptsize{}4. Threatening tone/Consequences}{\scriptsize \par}

{\scriptsize{}5. Spoofed website}\tabularnewline
\midrule 
Wetzel \citep{wetzel:2005}\tabularnewline
\midrule 
{\scriptsize{}1. Planning}{\scriptsize \par}

{\scriptsize{}2. Setup}{\scriptsize \par}

{\scriptsize{}3. Attack}{\scriptsize \par}

{\scriptsize{}4. Collection}{\scriptsize \par}

{\scriptsize{}5. Fraud}{\scriptsize \par}

{\scriptsize{}6. Post-attack}\tabularnewline
\midrule 
Tally, et al. \citep{tally:2004}\tabularnewline
\midrule 
{\scriptsize{}1. The attacker obtains E-mail addresses for the intended
victims}{\scriptsize \par}

{\scriptsize{}2. The attacker generates an E-mail that appears legitimate}{\scriptsize \par}

{\scriptsize{}3. The attacker sends the E-mail to the intended victims
in a way that appears legitimate and obscures the true source}{\scriptsize \par}

{\scriptsize{}4. The recipient opens a malicious attachment, completes
a form, or visits a web site}{\scriptsize \par}

{\scriptsize{}5. Harvest and exploitation}\tabularnewline
\midrule 
Emigh \citep{emigh:2005}\tabularnewline
\midrule 
{\scriptsize{}1. A malicious payload arrives through some propagation
vector}{\scriptsize \par}

{\scriptsize{}2. The user takes an action that makes him or her vulnerable
to an information compromise}{\scriptsize \par}

{\scriptsize{}3. The user is prompted for confidential information,
either by a remote web site or locally by a Web Trojan}{\scriptsize \par}

{\scriptsize{}4. The user compromises confidential information}{\scriptsize \par}

{\scriptsize{}5. The confidential information is transmitted from
a phishing server to the phisher}{\scriptsize \par}

{\scriptsize{}6. The confidential information is used to impersonate
the user}{\scriptsize \par}

{\scriptsize{}7. The phisher engages in fraud using the compromised
information}\tabularnewline
\midrule 
Nero et al. \citep{nero:2011}\tabularnewline
\midrule 
{\scriptsize{}1. Preparation }{\scriptsize \par}

{\scriptsize{}2. Delivery of the Lure }{\scriptsize \par}

{\scriptsize{}3. Taking the Bait }{\scriptsize \par}

{\scriptsize{}4. Request for Confidential Information }{\scriptsize \par}

{\scriptsize{}5. Submission of Information }{\scriptsize \par}

{\scriptsize{}6. Collection of Data }{\scriptsize \par}

{\scriptsize{}7. Impersonation }{\scriptsize \par}

{\scriptsize{}8. Financial Gain}\tabularnewline
\midrule
\end{longtable}
\par\end{flushleft}

Based on the example scenario explained earlier, phishing attacks
may consist of several phases. J. Hong \citep{hong:2012} argued that
there are three major phases. while Frauenstein, et al. \citep{frauenstein:2013}
suggested that there are five main processes are used to perform phishing
attacks based on the perspective of the attacker.

\begin{figure}
\begin{centering}
\includegraphics[scale=0.5]{\string"../NurulAkbarThesis-LyX/classicthesis-LyX-v4.1/gfx/phishing processes\string".png}
\par\end{centering}

\protect\caption{\label{fig:frauenstein}Phishing processes based on Frauenstein\citep{frauenstein:2013}}
\end{figure}


As we illustrated in Figure \ref{fig:frauenstein}, on the first process
is called \textit{Planning}, a phisher usually would do some reconnaissance
on how would the attack is executed and what information would be
obtained from the victim. On the second process, the phisher would
think about the design of the email. This email is desired by the
phisher to look as legit as possible to potential victim. For this
purpose, target institutions logo, trademark, or symbol are used to
make the content looks official and legitimate to the victim. The
author called this process as \textit{Email Design}. Figure \ref{fig:ing}
illustrates the example design of a fake email that impersonates ING
bank that can trick unsuspecting victims%
\footnote{http://www.martijn-onderwater.nl/wp-content/uploads/2010/03/ing-phishing.jpg%
}. From the figure, we can spot a fake email by investigating the sender
email address or the URL provided in the body, whether it redirects
to the official ING website or not. On the third process, the phisher
\textit{fabricates} a story to make potential victim think that email
is important. To achieve users attention, phisher might build up a
story about system upgrade, account hijacked or security enhancement
so that the victim would feel obliged to be informed. Evidently, this
technique corresponds with Cialdini \citep{cialdini:2001} that suggests
there are six principles to persuade people to comply with a request.
On the fourth process, a phisher usually include \textit{threatening
tone} or explain the urgency and consequences if the potential victim
chooses not to take action desired by the phisher (for example; account
removal, account blocked, etc.). Consequently, users may fear of their
account being deleted. The last process involved with fraudulent website
that has been created by the phisher. Users may falsely believe to
the message given in the email and may click a URL that is embedded
in the email. Subsequently, the URL would redirect users to a \textit{spoofed
website} which may prompt users\textquoteright{} sensitive information.
Furthermore, the website might be created to be as similar as possible
to the target institution\textquoteright s website, so that potential
victim may still believe that it is authentic. We will explain more
on Cialdini\textquoteright s six basic tendencies of human behavior
in generating positive response to persuasion \citep{cialdini:2001}
in a later section.

\begin{figure}
\begin{centering}
\includegraphics[scale=0.4]{../NurulAkbarThesis-LyX/classicthesis-LyX-v4.1/gfx/ing-phishing}
\par\end{centering}

\protect\caption{\label{fig:ing}Example of a phishing email impersonating ING bank}
\end{figure}


Considering that phishing attack is a process, Wetzel \citep{wetzel:2005}
suggested a taxonomy to make sense of the complex nature of the problem
by mapping out a common attacks lifecycle, and a possible set of activities
attackers engage in within each phase. The taxonomy is illustrated
in Figure \ref{fig:wetzel}. We speculated that Wetzel's taxonomy
is not analogous with Frauenstein's main phishing processes \citep{frauenstein:2013}.
The difference is that Frauenstein et al. only focus in the design
of the attack while Wetzel has added several phases like \textit{Collection},
\textit{Fraud} and \textit{Post-attack}, therefore, Wetzel taxonomy
is more holistic in term of phishing.

\begin{figure}
\centering{}\includegraphics[scale=0.4]{../NurulAkbarThesis-LyX/classicthesis-LyX-v4.1/gfx/wetzel}\protect\caption{\label{fig:wetzel}Phishing attack taxonomy and lifecycle\citep{wetzel:2005}}
\end{figure}


We have listed Wetzel's taxonomy in Table \ref{tab:compilation-phases},
we explain more of the taxonomy as follows:
\begin{enumerate}
\item \textit{Planning}: Preparation carried out by the phisher before continue
to the next phase. Example activities include identifying targets
and victims, determine the method of the attack, etc.
\item \textit{Setup}: After the target, victim and the method are known,
the phisher would craft a platform where the victim's information
could be transmitted and stored, for example: fraudulent website/email.
\item \textit{Attack}: Phisher distributes their fraudulent platform so
that it can be delivered to the potential victims with fabricated
stories.
\item \textit{Collection}: Phisher collects valuable information via response
from the victims
\item \textit{Fraud}: Phisher abuses victim's information by impersonates
the identity of the victim to the target. For example, A has gained
B's personal information to access C so that A can pose as B to access
C.
\item \textit{Post-attack}: After the phisher gained profit from the attack
and abuse phases, a phisher would not want to be noticed or detected
by authority. Thus, phisher might need to destroy evidence of the
malicious activities that he/she committed.
\end{enumerate}
As shown in Table \ref{tab:compilation-phases}.Tally, et al. suggest
that there are several phases involved in phishing attack based on
the attacker's point of view \citep{tally:2004}. The first phase,
it represents the planning, as we understand the attacker collects
the email address of unsuspecting victims. The second phase, considering
that it is related to creating a fake email that appears legitimate,
this phase can be viewed as design phase. On the third phase, we consider
this as delivery and attack phases as it involves the attacker sends
the fake email to the unintended victims and obfuscated the true source.
The fourth phase represents attack phase as it involves with the recipient
complies with the attacker's request(s). Lastly the fifth phase, it
represents the fraud phase, as it related to harvesting and exploiting
victim's resources by the attacker. Additionally, the phases described
by Tally, et al. \citep{tally:2004} are comparable with the information
flow explained by Emigh\citep{emigh:2005} represented in Figure \ref{fig:emigh}
and explained in Table \ref{tab:compilation-phases}. Phishing attack
steps that executed by the phisher are also being addressed by Nero,
et al \citep{nero:2011}. In their study, a successful phishing attack
involves several phases which can bee seen and compare in Table \ref{tab:compilation-phases}.

\begin{figure}
\centering{}\includegraphics[scale=0.4]{../NurulAkbarThesis-LyX/classicthesis-LyX-v4.1/gfx/emigh}\protect\caption{\label{fig:emigh}Flow of information in phishing attack \citep{emigh:2005}}
\end{figure}


Based on our analysis by looking at the pattern of other phases from
various sources, there is a major similarity between them. Therefore,
we would like to define and design our own phase that are integrated
with three key components suggested by Jakobsson, et al. \citep{jakobsson:2006}.
These key components are include \textit{the lure},\textit{ the hook}
and \textit{the catch}. As we designed in Figure \ref{fig:Information-flow-phishing},
we synthesized these three components with our phases based on the
attacker point of view as follows:

- The lure 

1. Phishers prepare the attack 

2. Deliver initial payload to potential victim 

3. Victim taking the bait

- The hook 

4. Prompt for confidential information

5. Disclosed confidential information 

6. Collect stolen information 

- The catch

7. Impersonates victim 

8. Received pay out from the bank 

\begin{figure}
\centering{}Collect stolen information \includegraphics[scale=0.3]{../NurulAkbarThesis-LyX/classicthesis-LyX-v4.1/gfx/info-flow-nolie}\protect\caption{\label{fig:Information-flow-phishing}Information flow phishing attack}
\end{figure}


It is important to know that in the phase 3, there are different scenarios
such as; victim might be redirected to a spoofed website, victim may
comply to reply the email, victim may comply to open an attachment(s)
or victim may comply to call by phone. However, in Figure \ref{fig:Information-flow-phishing},
we have only illustrated the phases if the bait was using a spoofed
website as a method. 

We have reviewed various phases in phishing attack and from the review,
we have constructed our own phases. In the next section, a brief introduction
in respect to the types of phishing will be described. We believe
that the general understanding of phishing types will help our main
analysis to characterize phishing email properties.


\section*{Types of phishing}

In January 2014, 8300 patients data are being compromised in medical
company in the US \citep{adam:2014}. The data includes names, addresses,
date of birth and phone numbers were being stolen. Other than demographic
information, clinical information associated with this data was also
stolen, including social security numbers. In the April 2014, phishers
have successfully stolen US\$163,000 from US public school based on
Michigan \citep{ashley:2014}. It has been said that the email prompted
to transfer money is coming from the finance director of the school.
In March 2014, Symantec has discovered phishing attack aimed at Google
drive users \citep{teri:2014}. The attack was carried firstly with
incoming email asking for opening document hosted at Google docs.
Users that have clicked on the link are taken to fraudulent Google
login page prompted Google users credentials. Interestingly, the URL
seems very convincing because it hosted on Google secure servers.
We hypothesized that even more phishing incidents on financial area
as well, but sometimes the news is kept hidden due to creditability
reason. With this in mind, we believe fake websites might be hosted
in the network which has more phishing domain than other networks.
Subsequently, in the next section, we will discuss bad neighborhoods
on phishing.

One may ask, what type of phishing are these? What are the general
types of phishing relevant to our research? Evidently, based on the
cost of phishing attacks in \ref{sec:The-cost-of}, the threat of
phishing attacks is still alarming and might be evolving in the future
with more sophisticated technique of attacks. For this reason, it
might be useful to provide a brief insight on popular variants of
phishing that currently exist. We will briefly explain the types of
phishing which are the most relevant to our research based on Jakobsson,
et al. \citep{jakobsson:2006}. These types of phishing are strongly
related to the phishing definition that we used, considering phishing
is based on the act of deception by the phishers.


\subsection*{Phishing based on visual similarities}

Since all phishing is based on deception and social engineering, there
is a phishing scenario based on visual similarities. Typical scenario
of phishing based on visual similarities is to send a large amount
of illicit emails containing call to action asking recipients to click
embedded links \citep{jakobsson:2006}. These variations include cousin
domain attack. For example, legitimate PayPal website addressed as
www.paypal.com, this cousin domain attacks confuse potential victims
to believe that www.paypal-security.com is a subdivision of the legitimate
website due to identical looking addresses. Similarly, homograph attacks
create a confusion using similar characters to its addresses. For
example, www.paypal.com and www.paypa1.com, both addresses look the
same but on the second link, it uses \textquotedblleft 1\textquotedblright{}
instead of \textquotedblleft l\textquotedblright . 

Moreover, phishers may embed a login page directly to the email content.
This suggests the elimination of the need of end-users to click on
a link and phishers do not have to manage an active fraudulent website.
IP addresses are often used instead of human readable hostname to
redirect potential victim to phishing website and JavaScript is used
to take over address bar of a browser to make potential victims believe
that they are communicating with the legitimate institution. We will
also see few examples of malicious JavaScript on our preliminary analysis
section. 

Another type of deceptive phishing scheme is rock-phish attacks. They
held responsible for half a number of reported incidents worldwide
in 2005 \citep{moore:2007}. These attacks evade email filters by
utilizes random text and GIF images which contain the actual message.
Rock phish attacks also utilize a toolkit that capable to manage several
fraudulent websites in a single domain. Sometimes, deceptive phishing
schemes lead to installation of malware when users visit fraudulent
website and we will describe malware based phishing scheme in the
next section. 


\subsection*{Malware-based phishing}

Generally, malware based phishing refers to any type of phishing which
involves installing malicious piece of software onto users' personal
computer \citep{jakobsson:2006}. Subsequently, this malware is used
to gather confidential information from victims instead of spoofing
legitimate websites. This type of phishing incorporates malwares such
as keyloggers/screenloggers, web Trojans and hosts file poisoning.

In the next section, we will study on the general phishing countermeasures
in term of phishing detection and prevention.

\selectlanguage{english}%

\section*{Current countermeasures}

\selectlanguage{american}%
There are various types of phishing countermeasures that implemented
in different levels. Purkait has conducted an extensive research in
reviewing these countermeasures which are available up until 2012
and their effectiveness \citep{purkait}. He suggests that there is
a classification of phishing countermeasures in separate groups and
according to Purkait \citep{purkait}, these groups are listed as
follow:
\begin{itemize}
\item Stop phishing at the email level
\item Security and password management toolbars
\item Restriction list
\item Visually differentiate the phishing site
\item Two facto and multi channel authentication
\item Takedown, transaction anomaly detection, log files
\item Anti phishing training
\item Legal solution
\end{itemize}
In addition, Parmar, et al. suggests that phishing detection can be
classified into two types; user training approach and software classification
approach \citep{parmar:2014}. He illustrated a diagram and a table
that summarizes phishing detection as countermeasures in a broad view
\citep{parmar:2014}. They also argued the advantages and disadvantages
of each category \citep{parmar:2014}. However, as our research mainly
focuses in synthesizing phishing email with cialdini's six principles
of persuasion \citep{cialdini:2001}, we will briefly discuss phishing
countermeasures such as restriction list group (i.e. Phishtank), machine
learning approach (web-based phishing), properties or features in
a phishing email, and anti phishing training group (i.e PhishGuru).
In the last section of this chapter, we will explore the human factor
in phishing attacks, how phishing email is engineered to gain recipient's
trust in order to get a response from the unsuspecting victims.

\selectlanguage{english}%

\subsection*{Phishing detection}

\selectlanguage{american}%
In this subsection, we will conduct a literature review which related
to phishtank as restriction list and machine learning approach to
detect spoofed website as phishing detection.


\subsubsection*{Phishtank}

One of the common approaches to detect phishing attacks is the implementation
of restriction list. As the name suggest, it prevents users to visit
fraudulent websites. One of the efforts to achieve restriction list,
is to derive phishing URLs from Phishtank. Phishtank is a blacklisting
company specifically for phishing URLs and it is a free community
web based where users can report, verify and track phishing URLs \citep{phishtank}.
Phishtank stores phishing URLs in its database and is widely available
for use by other companies for creating restriction list. Some of
the big companies that are using Phishtank\textquoteright s data includes;
Yahoo Mail, McAfee, APWG, Web Of Trust, Kaspersky, Opera and Avira.
In this section, we will discuss how the current literatures have
to do with phish data provided by Phishtank. The first step to achieve
the list of relevant literatures regarding phishtank is by keyword
search in Scopus online library. By putting ``Phishtank'' as a keyword
search, it results in 12 literatures. The next step, we read the all
the abstracts and conclusions of the resulting keyword search and
we decided 11 literatures that are relevant to our research. Lastly,
Table \ref{tab:phishtank} summarizes the papers selected and its
relevancy with Phishtank

\begin{longtable}{>{\raggedright}p{3cm}>{\raggedright}p{2cm}>{\raggedright}p{5cm}}
\caption{\label{tab:phishtank}Summary phishtank studies}
\tabularnewline
\toprule 
\textbf{\footnotesize{}Paper title} & \textbf{\footnotesize{}First author} & \textbf{\footnotesize{}Relevancy with phishtank}\tabularnewline
\midrule 
{\scriptsize{}Evaluating the wisdom of crowds in assessing phishing
website \citep{moore:2008}} & {\scriptsize{}Tyler Moore} & {\scriptsize{}Examine the structure and outcomes of user participation
in Phishtank. The authors find that Phishtank is dominated by the
most active users, and that participation follows a power law distribution
and this makes it particularly susceptible to manipulation.}\tabularnewline
\midrule 
{\scriptsize{}Re-evaluating the wisdom of crowds in assessing web
Security \citep{chia:2012}} & {\scriptsize{}Pern Hui Chia} & {\scriptsize{}Examine the wisdom of crowds on web of trust that has
similarity with Phishtank as a user based system.}\tabularnewline
\midrule 
{\scriptsize{}Automatic detection of phishing target from phishing
webpage \citep{liu:2010}} & {\scriptsize{}Gang Liu} & {\scriptsize{}Phishtank database is used to test the phishing target
identification accuracy of their method.}\tabularnewline
\midrule 
{\scriptsize{}A method for the automated detection of phishing websites
through both site characteristics and image analysis \citep{white:2012}} & {\scriptsize{}Joshua S. White} & {\scriptsize{}Phishtank database is used to perform additional validation
of their method. They also collect data from twitter using twitter\textquoteright s
API to find malicious tweets containing phishing URLs}\tabularnewline
\midrule 
{\scriptsize{}Intelligent phishing detection and protection scheme
for online transaction \citep{barraclough:2013}} & {\scriptsize{}P.A. Barraclough} & {\scriptsize{}Phishtank features is used as one of the input of neuro
fuzzy technique to detect phishing website. The study suggested 72
features from Phishtank by exploring journal papers and 200 phishing
website.}\tabularnewline
\midrule 
{\scriptsize{}Towards preventing QR code based attacks on android
phone using security warning \citep{yao:2013}} & {\scriptsize{}Huiping Yao} & {\scriptsize{}Phishtank API is used for lookup whether the given QR
containing phishing URL in the Phishtank database.}\tabularnewline
\midrule 
{\scriptsize{}A SVM based technique to detect phishing URLs \citep{huang:2012}} & {\scriptsize{}Huajun Huang} & {\scriptsize{}Phishtank database is used as validation resulting 99\%
accuracy by SVM method, plus the top ten brand names in Phishtank
archive is used as features in SVM method.}\tabularnewline
\midrule 
{\scriptsize{}Socio technological phishing prevention \citep{gupta:2011}} & {\scriptsize{}Gaurav Gupta} & {\scriptsize{}Analyze the Phishtank verifiers (individual/organization)
to be used as anti phishing model.}\tabularnewline
\midrule 
{\scriptsize{}An evaluation of lightweight classification methods
for identifying malicious URLs \citep{egan:2011}} & {\scriptsize{}Shaun Egan} & {\scriptsize{}Indicating that lightweight classification methods achieves
an accuracy of 93\% to 96\% when trained data from Phishtank.}\tabularnewline
\midrule 
{\scriptsize{}Phi.sh/\$oCiaL: The phishing landscape through short
URLs \citep{chhabra:2011}} & {\scriptsize{}Sidharth Chhabra} & {\scriptsize{}Phishtank database is used to analyze suspected phish
that is done through short URLs.}\tabularnewline
\midrule 
{\scriptsize{}Discovering phishing target based on semantic link network
\citep{wenyin:2010}} & {\scriptsize{}Liu Wenyin} & {\scriptsize{}Phishtank database is used as test dataset to verify
their proposed method (Semantic Link Network) }\tabularnewline
\end{longtable}

From our literature survey, we know that Phishtank is crowd-sourced
platform to manage phishing URLs. For that reason Moore, et al. aims
to evaluate the wisdom of crowds platform accommodated by Phishtank
\citep{moore:2008}. Moore, et al. suggest that the user participation
is distributed according to power law. It uses to model data which
frequency of an event varies as a power of some attribute of that
event \citep{lai}. Power law also applies to a system when large
is rare and small is common %
\footnote{http://kottke.org/03/02/weblogs-and-power-laws%
}. For example, in the case of individual wealth in a country, 80\%
of the all wealth is controlled by 20\% of population in a country.
It makes sense that in Phishtank\textquoteright s verification system,
a single highly active user\textquoteright s action can greatly impact
the system\textquoteright s overall accuracy. Table \ref{tab:phishtank}
summarizes the comparison performed by \citep{moore:2008} between
Phishtank and closed proprietary anti-phishing feeds%
\footnote{The author conceals the identity of the closed proprietary company%
}. Moreover, there are some ways to disrupt Phishtank verification
system; submitting invalid reports accusing legitimate website, voting
legitimate website as phish, and voting illegitimate website as not
phish. While all the scenarios described are for the phishers' benefit,
the last scenario is more direct and the first two actions rather
subtle intention to undermine Phishtank credibility.

To put it briefly, the lesson of crowd sourced anti-phishing technology
such as Phishtank is that the distribution of user participation matters.
It means that if a few high value participants do something wrong,
it can greatly impact overall system \citep{moore:2008}. Also, there
is a high probability that bad users could also extensively participate
in submitting or verifying URLs in Phishtank.

\begin{table}
\begin{tabular}{|>{\centering}p{4cm}|>{\centering}p{3cm}|}
\hline 
\textbf{\scriptsize{}Phishtank} & \textbf{\scriptsize{}Proprietary}\tabularnewline
\hline 
\hline 
{\scriptsize{}10924 URLs} & {\scriptsize{}13318 URLs}\tabularnewline
\hline 
{\scriptsize{}8296 URLs after removing duplication} & {\scriptsize{}8730 URLs after removing duplication}\tabularnewline
\hline 
\multicolumn{2}{|c|}{{\scriptsize{}Shares 5711 URLs in common 3019 Unique to the company
feeds while 2585 only appeared in Phishtank}}\tabularnewline
\hline 
{\scriptsize{}586 rock-phish domains} & {\scriptsize{}1003 rock phish domains}\tabularnewline
\hline 
{\scriptsize{}459 rock phish domains found in Phishtank} & {\scriptsize{}544 rock phish domains not found in Phishtank}\tabularnewline
\hline 
{\scriptsize{}Saw the submission first} & {\scriptsize{}11 minutes later appear on the feed}\tabularnewline
\hline 
{\scriptsize{}16 hours later after its submission for verification
(voting based)} & {\scriptsize{}8 second to verified after it appears}\tabularnewline
\hline 
{\scriptsize{}Rock phish appear after 12 hours appeared in the proprietary
feed and were not verified for another 12 hours} & \selectlanguage{english}%
\selectlanguage{american}%
\tabularnewline
\hline 
\end{tabular}\protect\caption{\label{tab:Comparison-summary}Comparison summary \citep{moore:2008}}
\end{table}



\subsubsection*{Machine learning approach in detecting spoofed website}

The fundamental of phishing detection system would be to distinguish
between phishing websites and the legitimate ones. As we previously
discussed, the aim of phishing attack is to gather confidential information
from potential victims. To do this, phishers often prompt for this
information through fraudulent websites and masquerade as legitimate
institutions. It does not make sense if phishers created them in a
way very distinctive with its target. It may raise suspicions with
result of unsuccessful attack. To put it another way, while it might
be true, we speculated that most of the phishing websites are mostly
identical with its legitimate websites as target to reduce suspiciousness
from potential victim. 

In contrast of one of blacklisting technique we saw in Phishtank that
heavily depend on human verification, researchers make use of machine
learning based technique to automatically distinguish between phishing
and legitimate either websites or email. Basically, machine-learning
system is a platform that can learn from previous data and predict
future data with its classification, in this case, phishing and legitimate.
In order for this machine to learn from data, there should be some
kind of inputs to classify the data, it is called features or characteristics. 

Furthermore, there are also several learning algorithms to classify
the data, such as, logistic regression, random forest, neural networks
and support vector machine. However, as this particular topic is out
of scope of our research, we will not discuss about the learning algorithm
that is currently implemented. We will only introduce three features
that are used in machine learning based detection. 

There are vast amount of features to be used in machine learning to
detect phishing attack. Literatures are selected by keyword search
such as ``phishing + detection + machine learning''. We analyze
three features: lexical feature, host-based feature and site popularity
feature. Each of these features will be introduced briefly as follows.
\begin{itemize}
\item Lexical features
\end{itemize}
Lexical features (URL based features) are based on the analysis of
URL structure without any external information. Ma, et al. suggest
that the structure URL of phishing may \textquotedblleft look\textquotedblright{}
different to experts \citep{ma:2009}. These features include how
many dots exist, the length, how deep the path traversal do the URL
has or if there any sensitive words present in a URL. For example
the URLs https://www.paypal.com and http://www.paypal.com.example.com/
or http://login.example.com/ www.paypal.com/, we can see that the
domain paypal.com positioned differently, with the first one being
the benign URL. Le, et al suggests we can extract the features related
to the full URL, domain name, directory, file name and argument \citep{le:2011}.
For example we want to extract features related to the full URL; we
can define the length of the URL, the number of dots in the URL, and
whether the blacklisted word presents in the URL. The blacklisted
words consist of sensitive words such as confirm, account, login or
webscr. 

Lexical features analysis may have performance advantage and reduces
overhead in term of processing and latency, since it only tells the
machine to learn URL structure. 90\% accuracy is achieved when utilizing
lexical features combined with external features such as WHOIS data
\citep{le:2011}. Egan, et al. conducted an evaluation of lightweight
classification that includes lexical features and host based features
in its model \citep{egan:2011}. The study found that the classification
based on these features resulted in extremely high accuracy and low
overhead. Table \ref{tab:exist-lex} lists the existing lexical features
that are currently implemented by two different studies \citep{xiang:2011,liu}.
However, Xiang, et al.\citep{xiang:2011} pointed out that URLs structure
could be manipulated with little cost, causing the features to fail.
For example, attackers could simply remove embedded domain and sensitive
words to make their phishing URLs look legitimate. Embedded domain
feature examines whether a domain or a hostname is present in the
path segment \citep{xiang:2011}, for example, http://www.example.net/pathto/www.paypal.com.
Suspicious URL feature examine whether the URL has ``@'' or ``-'',
the present of ``@'' is examined in a URL because when the symbol
``@'' is used, the string to the left will be discarded. Furthermore,
according to \citep{xiang:2011}, not many legitimate websites use
``-'' in their URLs. There are also plenty of legitimate domains
presented only with IP address and contains more dots. Nevertheless,
lexical analysis would be suitable features to use for first phase
analysis in a large data \citep{egan:2011}.

\begin{table}
\centering{}%
\begin{tabular}{|>{\raggedright}p{5cm}|>{\raggedright}p{3.5cm}|}
\hline 
\textbf{\scriptsize{}Haotian Liu, et al. \citep{liu}} & \textbf{\scriptsize{}Guang Xiang, et al. \citep{xiang:2011}}\tabularnewline
\hline 
\hline 
{\scriptsize{}- Length of hostname Length of entire URL}{\scriptsize \par}

{\scriptsize{}- Number of dots }{\scriptsize \par}

{\scriptsize{}- Top-level domain}{\scriptsize \par}

{\scriptsize{}- Domain token count}{\scriptsize \par}

{\scriptsize{}- Path token count}{\scriptsize \par}

{\scriptsize{}- Average domain token length of all dataset}{\scriptsize \par}

{\scriptsize{}- Average path token length of dataset}{\scriptsize \par}

{\scriptsize{}- Longest domain token length of dataset}{\scriptsize \par}

{\scriptsize{}- Longest path token length of dataset}{\scriptsize \par}

{\scriptsize{}- Brand name presence }{\scriptsize \par}

{\scriptsize{}- IP address presence}{\scriptsize \par}

{\scriptsize{}- Security sensitive word presence } & {\scriptsize{}- Embedded domain}{\scriptsize \par}

{\scriptsize{}- IP address presence}{\scriptsize \par}

{\scriptsize{}- Number of dots}{\scriptsize \par}

{\scriptsize{}- Suspicious URL }{\scriptsize \par}

{\scriptsize{}- Number of sensitive words}{\scriptsize \par}

{\scriptsize{}- Out of position top level domain (TLD) }\tabularnewline
\hline 
\end{tabular}\protect\caption{\label{tab:exist-lex}Existing lexical features \citep{liu,xiang:2011}}
\end{table}

\begin{itemize}
\item Host based features
\end{itemize}
Since phishers often hosted phishing websites in less reputable hosting
services and registrars, host-based features are needed to observe
on the external sources (WHOIS information, domain information, etc.).
A study suggests host-based features have the ability to describe
where phishing websites are hosted, who owns them and how they are
managed \citep{ma:2009}. Table \ref{tab:host-based} shows the host-based
features from three studies that are currently used in machine learning
phishing detection. These studies are selected only for example comparison.

\begin{table}
\centering{}%
\begin{tabular}{>{\raggedright}p{2cm}>{\raggedright}p{4cm}>{\raggedright}p{2cm}}
\toprule 
\textbf{\scriptsize{}Justin Ma, et al.\citep{ma:2009,ma2:2009}} & \textbf{\scriptsize{}Haotian Liu, et al. {[}46{]}\citep{liu}} & \textbf{\scriptsize{}Guang Xiang, et al. \citep{xiang:2011}}\tabularnewline
\midrule
\midrule 
{\scriptsize{}- WHOIS data}{\scriptsize \par}

{\scriptsize{}- IP address information}{\scriptsize \par}

{\scriptsize{}- Connection speed}{\scriptsize \par}

{\scriptsize{}- Domain name properties } & {\scriptsize{}- Autonomous system number }{\scriptsize \par}

{\scriptsize{}- IP country}{\scriptsize \par}

{\scriptsize{}- Number of registration information}{\scriptsize \par}

{\scriptsize{}- Number of resolved IPs}{\scriptsize \par}

{\scriptsize{}- Domain contains valid PTR record}{\scriptsize \par}

{\scriptsize{}- Redirect to new site}{\scriptsize \par}

{\scriptsize{}- All IPs are consistent} & {\scriptsize{}- Age of Domain}\tabularnewline
\bottomrule
\end{tabular}\protect\caption{\label{tab:host-based}Host-based features \citep{ma:2009,ma2:2009,liu,xiang:2011}}
\end{table}


Each of these features does matter for phishing detection. However,
as our main objective is synthesizing cialdini's principle with phishing
emails, we will not describe each of these features in detail. It
is noteworthy that some of the features are subset of another feature,
for instance, autonomous system number (ASN), IP country and number
of registration information are derived from WHOIS information. Nevertheless,
we will only explain few of them that we assume the most crucial. 
\begin{enumerate}
\item WHOIS information: Since phishing websites and hacked domains are
often created at relatively young age, this information could provide
the registration date, update date and expiration date. Domain ownership
would also be included; therefore, a set of malicious websites with
the same individual could be identified. 
\item IP address information: Justin Ma, et al. used this information for
identify whether or not an IP address is in blacklist \citep{ma2:2009,ma:2009}.
Besides the corresponding IP address, it provides records like nameservers
and mail exchange servers. This allows the classifier to be able to
flag other IP addresses within the same IP prefix and ASN. 
\item Domain name properties: these include time to live (TTL) of DNS associated
with a hostname. PTR record (reverse DNS lookup) of a domain could
also be derived whether it is valid or not.\end{enumerate}
\begin{itemize}
\item Site popularity features
\end{itemize}
Site popularity could be an indicator whether a website is phishy
or not. It makes sense if a phishing website has much less traffic
or popularity than a legitimate website. According to \citep{xiang:2011},
some of the features indicated in Table \ref{tab:popular-features}
are well performed when incorporated with machine learning system. 

\begin{table}
\centering{}%
\begin{tabular}{>{\raggedright}p{5cm}>{\raggedright}p{3.5cm}}
\toprule 
\textbf{\footnotesize{}Guang Xiang, et al. \citep{xiang:2011}} & \textbf{\footnotesize{}Haotian Liu, et al. \citep{liu}}\tabularnewline
\midrule
\midrule 
{\scriptsize{}- Page in top search results }{\scriptsize \par}

{\scriptsize{}- PageRank}{\scriptsize \par}

{\scriptsize{}- Page in top results when searching copyright company
name and domain}{\scriptsize \par}

{\scriptsize{}- Page in top results when searching copyright company
name and hostname } & {\scriptsize{}- Number of external links}{\scriptsize \par}

{\scriptsize{}- Real traffic rank}{\scriptsize \par}

{\scriptsize{}- Domain in reputable sites list}\tabularnewline
\bottomrule
\end{tabular}\protect\caption{\label{tab:popular-features}Site popularity features \citep{xiang:2011,liu}}
\end{table}

\begin{enumerate}
\item Page in top search results: this feature originally used by \citep{zhang:2007}
to find whether or not a website shows up on the top N search result.
If it is not the case, the website could be flagged as phishy since
phishing websites have less chance of being crawled \citep{xiang:2011}.
We believe this feature is similar to Number of external links feature
since both of them are implying the same technique.
\item PageRank: this technique is originally introduced by Google to map
which websites are popular and which are not, based on the value from
0 to 10. According to \citep{xiang:2011}, the intuitive rationale
of this feature is that phishing websites are often have very low
PageRank due to their ephemeral nature and very low incoming links
that are redirected to them. This feature similar to Real traffic
rank feature employed by \citep{liu} where such feature can be acquired
from alexa.com.
\item Page in top results when searching copyright company name and domain/hostname
features are complement features of Page in top search results feature
with just different queries. Moreover, we believe they are also similar
to Domain in reputable sites list feature since they are determining
the reputation of a website. The first two features can be identified
by querying google.com \citep{xiang:2011} and the latter feature
can be obtained from amazon.com \citep{liu}. 
\end{enumerate}

\subsubsection*{Stop phishing at email level}

In order to stop phishing at email level, phishing email properties
or features should be investigated. Chandrasekaran, et al. and Drake,
et al \citep{chandrasekaran:2006,drake2004anatomy} specify the structure
of phishing emails properties as follows:
\begin{enumerate}
\item Spoofing of online banks and retailers. Impersonation of legitimate
institutions may created in the email level. Phishers may design a
fake email to resemble the reputable company to gain users trust.
\item Link in the text is different from the destination. A link(s) contained
in the email message usually appear to different than the actual link
destination. This method used to trick users to believe that the email
is legitimate.
\item Using IP addresses instead of URLs. Sometimes phishers may hide the
link in the message by presenting it as IP address instead of URL.
\item Generalization in addressing recipients. As phishing emails are distributed
by large number of recipients, the email often is not personalized,
unlike the legitimate email that address its recipient by personalized
information such as the last four digits of account information.
\item Usage of well-defined situational contexts to lure victims. Situational
contexts such as false urgency and threat are a common method to influence
the decision making of the recipients.
\end{enumerate}
Moreover, Ma, et al. experimented with seven properties to consider
in a phishing emails consist of the total number of links, total numbers
of invisible links, whether the link that appears in the message is
different than the actual destination, the existence of forms, whether
scripts exist within an email, total appearance of blacklisted words
in the body and the total appearance of blacklisted words in the subject
\citep{ma2009detecting}. Based on this survey, we will be establishing
phishing email properties as variables in order to classify our data.


\subsection*{Phishing prevention}

Phishing attacks aim to by-pass technological countermeasures by manipulating
users\textquoteright{} trust and can lead to monetary losses. Therefore,
human factors take a big part on the phishing taxonomy, especially
in the organizational environment. Human factor in phishing taxonomy
comprised of education, training and awareness \citep{frauenstein:2013}.
Figure \ref{fig:holistic} illustrates where human factor takes part
on phishing threats \citep{frauenstein:2013}. User\textquoteright s
awareness of phishing has been explored by several studies \citep{james:2005,frauenstein:2013,emigh:2005,kumaraguru:2008,jansson:2013,dodge:2007}
as preventive measure against phishing attack. According to ISO/IEC
27002 \citep{frauenstein:2013}\citep{organization:2005}, it has
been shown that information security awareness is important and it
has been critical success factors to mitigate security vulnerabilities
that attack user\textquoteright s trust. One approach to hopefully
prevent phishing attack was by implementing anti phishing warning/indicator.
Dhamija, et al suggest that users often ignore security indicators
thus makes them ineffective \citep{dhamija2006phishing}. Even if
users notice the security indicators, they often do not understand
what they represent. 

Moreover, the inconsistency of positioning on different browsers makes
them much difficult to identify phishing \citep{kirlappos:2012}.
Evidently, Schechter, et al. pointed out that 53\% of their study
participants were still attempting to provide their confidential information,
even after their task was interrupted by strong security warning \citep{schechter:2007}.
Therefore, these suggest that an effective phishing education must
be added as a complementary strategy to complete technical anti-phishing
measure as a strong remedy to detect phishing websites or emails.

\begin{figure}[H]
\centering{}\includegraphics[scale=0.6]{\string"../NurulAkbarThesis-LyX/classicthesis-LyX-v4.1/gfx/human factor\string".png}\protect\caption{\label{fig:holistic}Holistic anti-phishing framework \citep{frauenstein:2013}}
\end{figure}


Phishing education for online users often by instructing not to click
links in an email, ensure that SSL is present and to verify that the
domain name is correct before giving information, and other similar
education. This traditional practice evidently has not always effective
\citep{emigh:2005}. One may ask what makes phishing education effective?
A study suggests that in order online users to be aware of phishing
threats, is to really engage them to so that they understand how vulnerable
they are \citep{mansfield:2013}. To do this, simulated phishing attacks
often performed internally in an organization. Figure \ref{fig:simulated}
shows a simulated phishing email and website carried out by Kumaraguru,
et al. from PhishGuru \citep{kumaraguru:2009}. As a result, this
scenario puts them in the ultimate teachable moment if they fall for
these attacks.

\begin{figure}[H]
\subfloat[simulated phishing email \citep{kumaraguru:2009}]{\centering{}\includegraphics[scale=0.6]{../NurulAkbarThesis-LyX/classicthesis-LyX-v4.1/gfx/kumaraguru-a}}

\quad{}\subfloat[simulated phishing website \citep{kumaraguru:2009}]{\centering{}\includegraphics[scale=0.6]{../NurulAkbarThesis-LyX/classicthesis-LyX-v4.1/gfx/kumaraguru-b}}\quad{}\subfloat[simulated phishing message \citep{kumaraguru:2009}]{\centering{}\includegraphics[scale=0.6]{../NurulAkbarThesis-LyX/classicthesis-LyX-v4.1/gfx/kumaraguru-c}}\protect\caption{\label{fig:simulated}Simulated phishing attack \citep{kumaraguru:2009}}
\end{figure}


Phishguru is a security training system operated by Wombat security
technology that teaches users not to be deceived by phishing attempts
by simulation of phishing attacks\citep{phishguru}. They claimed
Phishguru provides more effective training than traditional training
as it is designed to be more engaging. Figure \ref{fig:em-training}
illustrates how embedded phishing training was presented by PhishGuru.

Kumaraguru, et al. investigates the effectiveness of embedded training
methodology in a real world situation \citep{kumaraguru:2009}. Evidently,
they indicated that even after 28 days after training, users trained
by PhishGuru were less likely to click the link presented in the simulated
phishing email than those who were not trained. They also find that
users who trained twice were less likely to give information to simulated
fraudulent website than users who were trained once. Moreover, they
argue that the training does not decrease the users\textquoteright{}
willingness to click on the links from legitimate emails; it means
that less likely a trained user did a false positive when he or she
requested to give information from true legitimate emails \citep{kumaraguru:2009}.
This suggests that user training strategy as an effective phishing
education in order to improve phishing awareness especially in organizational
environment.

\begin{figure}[H]
\centering{}\includegraphics[scale=0.6]{\string"../NurulAkbarThesis-LyX/classicthesis-LyX-v4.1/gfx/phishing training\string".png}\protect\caption{\label{fig:em-training}Embedded phishing training \citep{kumaraguru:2009}}
\end{figure}



\section*{Human factor}

Phishing attacks generally aim to manipulate end users to comply phisher's
request. Such manipulation in phishing attacks is achieved by social
engineering. It means that human element is tightly associated with
phishing. But how do phishers compose such deception? How come online
users are gullible to these attacks? 

Kevin Mitnick, who was obtaining millions of dollars by performing
social engineering technique, is plausibly the best known person who
had used social engineering technique to carry out his attacks. His
book that titled ``The art of deception: Controlling the Human Element
of Security'' \citep{mitnik:2001} has defined social engineering
as follows:
\begin{quote}
``Using influence and persuasion to deceive people by convincing
them that the attacker is someone he is not, or by manipulation. As
a result, the social engineer is able to take advantage of people
to obtain information, or to persuade them to perform an action item,
with or without the use of technology.''

From his definition we can learn that people are the main target of
the attack, specifies some of the important tools used by the attackers,
such as influence and persuasion. 
\end{quote}
Cialdini suggests that there are six basic principles of persuasion
\citep{cialdini:2001}, that is, the technique of making people grant
to one's request. These principles include;\textit{ reciprocation,
consistency, social proof, likeability, authority and scarcity}. Reciprocation
is \foreignlanguage{english}{the norm that obligates individuals to
repay in kind what they have received, return the favor or adjustment
to smaller request \citep{cialdini:2001}. Consistency is a public
commitment where people become psychologically become vested in a
decision they have made \citep{workman:2008}\citep{cialdini:2001}.
Social proof is when people model the behavior of their peer group,
role models, important others or because it is generally \textquotedbl{}fashionable\textquotedbl{}
\citep{workman:2008}. Stajano, et al. suggest people will let their
guard down when everybody around them appears to share the same risk
\citep{stajano2011understanding}. Likeability is when people trust
and comply with requests from others who they find attractive or having
credibility \citep{workman:2008,cialdini:2001}. While it is our human
nature not to question authority, it can be used to engender fear,
where people obey commands to avoid negative consequences such as
losing a privilege or something of value, punishment, humiliation
or condemnation \citep{cialdini:2001,workman:2008}. Stajano, et al
suggest that scarcity is related to time principle, that is, when
we are under time pressure to make important choice, we tend to have
less reasoning to make decision \citep{stajano2011understanding}.
We will use these principles as our foundation in synthesizing phishing
email corpus with human factor. }

Human as the ``weakest link'' in computer security has been exists
and exploited for ages. And yet, security designers blame on users
and whine ``the system I designed would be secure, if only users
were less gullible'' \citep{stajano2011understanding}. Stajano,
et al. stated that ``a wise security designer would seek a robust
solution which acknowledge the existence of these vulnerabilities
as unavoidable consequence of human nature and actively build countermeasures
that prevent this exploitation'' \citep{stajano2011understanding}.
With this in mind, the exploration of persuasion principles is congruent
with our research goal. Cialdini's six persuasion principles will
be the foundation in our research.


\part*{Research questions and Hypotheses}

One of the primary objectives of our literature reviews is to summarize
the existing literature that explain the general topic about phishing
and formulate the research questions in the process. We continue this
report by enumerating our research questions and also mentioning our
hypotheses related to these questions. We aim to test these hypotheses
by the data collected during our study of phishing emails provided
by fraudehelpdesk.nl. Firstly, we wanted to know the characteristics
of phishing email based on structural properties in our corpus.

\selectlanguage{english}%
\rule[0.5ex]{1\columnwidth}{1pt}

\textit{RQ1: What are the characteristics of phishing emails?}

\rule[0.5ex]{1\columnwidth}{1pt}

The characteristics of phishing emails in our dataset are determined
by the following parameters:
\begin{itemize}
\item How often phishing email include an attachment(s) and what specific
attachment is the most frequent.
\item Prevalent methods
\item Content characteristics
\item The most targeted institutions
\item The reasons that are frequently being used
\item Persuasion principles characteristics
\item Relationship between generic properties
\end{itemize}
Secondly, we wanted to know how relevant are the persuasive principles
to the associated phishing eamil properties. 

\textit{\rule[0.5ex]{1\columnwidth}{1pt}}

\textit{RQ2: How relevant are the persuasive principles to the generic
phishing email properties?}

\rule[0.5ex]{1\columnwidth}{1pt}

We established 16 hypotheses to indicate the relationship between
generic properties and relevancy of persuasive principle to these
properties. H8, H9, H10, H13, H14, H15 will partly answer RQ1 in respect
to the relationship between generic properties and the rest will answer
RQ2. We synthesize cialdini's principles with our dataset. In order
to conduct synthesization, we established our decision making to classify
which persuasive elements that are exist in a phishing email. 

In our coding of cialdini's principles and phishing email dataset,
we identified phishing emails with fake logos and signatures that
may mistakenly regard them as legitimate by average internet users.
For example in the context of phishing email, signature such as ``Copyright
2013 PayPal, Inc. All rights reserved'' or ``Administrator Team''
and Amazon logo were used to show the ``aura of legitimacy''. In
the real world society, telemarketers and seller has been using authoritative
element to increase the chance of potential consumer's compliance
\citep{telemarket:2013}. It means that they have to provide information
in a confident way. Consumers will have their doubt if sellers unsure
and nervous when they offer their product and services to consumers.
This principle has been one of the strategies in a social engineering
attack to acquire action and response from a target \citep{npdn:2013}.

It is makes sense if government has the authority to compose laws
and regulations and to control its citizens. Government sector includes
court and police department also authorize to execute penalties if
any wrongdoing happens within their jurisdiction. However, government
may not have to be likeable to enforce their rules and regulation.
Similarly, an administrator who control his network environment may
behave in a similar fashion as government. Hence, in our dataset we
hypothesize that

\rule[0.5ex]{1\columnwidth}{1pt}

\textit{H1: There will be a significant association between Government
sector and authority principle}

\textit{H2: Phishing emails which targeting Administrator will likely
to have authority principle }

\rule[0.5ex]{1\columnwidth}{1pt}

Similar to authority principle that may trigger reactance, scarce
items and shortage may produce immediate compliance from people. In
essence, people will react when their freedom is restricted about
valuable matter when they think they are capable to make a choice
among different options \citep{pennebaker1976american}. For example
in phishing email context, an email from Royal Bank inform us that
we have not been logged into our online banking account for a quite
some time, as a security measure, they must suspend our online account
and if we would like to continue to use the online banking facility,
we have been asked to click the URL provided. Potential victim may
perceives their online banking account as their valuable matter to
access facility and information about their savings. Consequently,
potential vicim may react to the request because of their account
could be scarce and restricted. In the real world example, a hard
worker bank customer who perceives money is a scarce item may immediately
react when his bank inform him that he is in danger of losing his
savings due to ``security breach''. We therefore hypothesize that

\rule[0.5ex]{1\columnwidth}{1pt}

\textit{H3: There will be a significant correlation between Financial
sector and scarcity principle}

\rule[0.5ex]{1\columnwidth}{1pt}

As we describe in our decision making consideration section, people
tend to trust those they like. In a context of persuasion, perpetrators
may find it more difficult to portray physical attractiveness, instead
they are relying on emails, websites and phone calling \citep{dotterweich2006practicality}.
To exhibit charm or charisma to the potential victims, perpetrators
may gain their trust by establishing friendly emails, affectionate
websites and soothing voice over the phone. In the phishing email
context, Amazon praises our existence in an appealing fashion and
extremely values our account security so that no one can break it.
Based on this scenario, E-commerce/Retails sector may applied likeability
principles to gain potential customers. We therefore hypothesize that

\rule[0.5ex]{1\columnwidth}{1pt}

\textit{H4: Phishing emails which targeting E-Commerce/Retails will
likely to have a significant relationship with likeability principle}

\rule[0.5ex]{1\columnwidth}{1pt}

Tajfel, et al. argued that people often form their own perception
based on their relationship with others in a certain social circles
\citep{tajfel2004social}. This lead to affection of something when
significant others have something to do with it. Social proof is one
of the social engineering attacks based on the behavioral modeling
and conformance \citep{workman:2008}For example, we tend to comply
to a request when a social networking site asks us to visit a website
or recommends something and mention that others have been visiting
the website as well. Thus, we hypothesize that 

\rule[0.5ex]{1\columnwidth}{1pt}

\textit{H5: Phishing emails which targeting Social networks will likely
to have signification association with social proof principle }

\rule[0.5ex]{1\columnwidth}{1pt}

As we describe in our decision making consideration section, authority
has something to do with ``aura of legitimacy''. This principle
may lead to suggest the limitation on something that we deemed valuable.
For example, a perpetrator masquerades as an authority and dressed
as police officer halted us on the road, the perpetrator may tell
us that we did something wrong and he will held our driving license
if we do not pay him the fine. In the phishing email context, an email
masquerades as ``System Administrator'' may tell us that we exceeded
our mailbox quota, so the administrator must freeze our email account
and we could re-activate it by clicking the URL provided in the email.
Based on this scenario, we know that it has authority principle and
also has scarcity principle. Therefore, we hypothesize that

\rule[0.5ex]{1\columnwidth}{1pt}

\textit{H6: There will be a significant relationship between authority
principle and scarcity principle}

\rule[0.5ex]{1\columnwidth}{1pt}

We often stumbled a group of people requesting to donate some of our
money to the unfortunate people. Evidently, they would use physical
attractiveness and kind words to get our commitment to support those
people. Once they have got our commitment, they start asking for donation
and we tend to grant their request and give some of our money to show
that we are committed. Phishing email could be similar, for example,
Paypal appreciates our membership on their system and PayPal kindly
notifies us that in the membership term of agreement, they would performing
annual membership confirmation from its customers. Based on this scenario,
we know that the email has likeability principle and also has consistency
principle. We would like to know if it is the case with phishing email
in our dataset. Therefore, we hypothesize that

\rule[0.5ex]{1\columnwidth}{1pt}

\textit{H7: Phishing emails which have likeability principle will
likely to have consistency principle }

\rule[0.5ex]{1\columnwidth}{1pt}

We think it make sense if a fraudster tries to make his fake product
as genuine as possible and hide the fabricated element of his product.
There are also fraudster that did not make his product as identical
as the legitimate product. In the phishing email context, we perceives
fake product as URL in the email, phishers do not necessarily obfuscates
the real URL with something else. Logically, such phishers do not
aim to make a high quality of bogus email, rather they aim to take
chances in getting potential victims that are very careless. This
leads to our hypothesis that say

\rule[0.5ex]{1\columnwidth}{1pt}

\textit{H8: Phishing emails that include URL will likely to be different
than the real destination}

\rule[0.5ex]{1\columnwidth}{1pt}

It is conspicuous from our knowledge if a sales agent tries to sell
us a product, it would be followed by the request element to buy the
product as well. However, it will not make sense if he tries to sell
his product but he requests to buy another company's product. In other
words, if we have something to sell, we do not just display our product
without asking people's attention to look at our product. For example
in phishing email context, phishers may include URL or attachment
in the body of the email and also they may request unsuspecting victim
to click the URL or to open the attachment. This leads us to two hypotheses
which state

\rule[0.5ex]{1\columnwidth}{1pt}

\textit{H9: Phishing emails that include URL will likely to request
to click the URL}

\textit{H10: Phishing emails that include attachment will likely to
request to open the attachment}

\rule[0.5ex]{1\columnwidth}{1pt}

We sometimes find it suspicious if a person dressed as police officer
that does not have a badge carried with him, unless he is a fake police
officer. Consequently, a fake police officer may use a fake badge
to build up even more ``aura of legitimacy''. Evidently, Cialdini
suggests the increment of passerby who have stop and stare at the
sky by 350 percent with suit and tie instead of casual dress \citep{cialdini:2001}.
Hence, we correlate that a person who wears police uniform and a fake
badge in the real world context as authority principle and the presence
of image in the phishing mail context. Another example, an email that
masquerades Apple company, may clone Apple company logo or trademark
to its content to increase the chance of potential victim's response
or increase the |believability'' if you will. Thus, we hypothesize
that

\rule[0.5ex]{1\columnwidth}{1pt}

\textit{H11: Phishing emails that have authority principle will likely
to include an image to its content}

\rule[0.5ex]{1\columnwidth}{1pt}

Apart from the target analysis, we also investigate the reason why
potential victim responds to phisher's request. Phishing email that
implies our account expiration would have scarcity principle because
the account itself may very valuable for us and is in danger to be
expired or terminated. Therefore, we hypothesize that

\rule[0.5ex]{1\columnwidth}{1pt}

\textit{H12: There will be a significant association between account
related reason and scarcity principle }

\rule[0.5ex]{1\columnwidth}{1pt}

Similar from the hypothesis H12, it is sensible if a phishing email
which contains account related reason such as reset password or security
update, may tend to have a URL for the potential victim to be redirected
towards phisher's bogus website or malware. Regardless of the target,
based on our initial coding of the dataset we found that account related
reason in a phishing email needs an immediate action greater than
other reasons. Therefore, phishers may likely to include a URL to
have immediate response from the potential victim. This leads to our
hypothesis that say

\rule[0.5ex]{1\columnwidth}{1pt}

\textit{H13: Phishing emails which have account related reason will
likely to have URL }

\rule[0.5ex]{1\columnwidth}{1pt}

When a phishing email has document related reason such as review some
document reports or court notice, it may tend to impersonate government
to make the email sensible enough to persuade potential victim more
than other targets. We therefore hypothesize that

\rule[0.5ex]{1\columnwidth}{1pt}

\textit{H14: Phishing emails which targeting government sector will
likely to have document related reason }

\rule[0.5ex]{1\columnwidth}{1pt}

Analogous with the hypothesis \textit{H14}, it is make sense if a
phishing email which has document related reason such as reviewing
contract agreement or reviewing resolution case, would tend to have
a file to be attached. We therefore hypothesize that

\rule[0.5ex]{1\columnwidth}{1pt}

\textit{H15: Phishing emails which have document related reason will
likely to include attachment}

\rule[0.5ex]{1\columnwidth}{1pt}

We think it is make sense if a phishing email which use HTML to present
their email design may tend to increase the attractiveness to the
potential victim. Consequently, unsuspected victim may respond to
the request just because of the email design is attractive. Therefore,
we hypothesize that

\rule[0.5ex]{1\columnwidth}{1pt}

\textit{H16: Phishing emails which use HTML will have a significant
association with likeability principle}

\rule[0.5ex]{1\columnwidth}{1pt}


\part*{Research methodology}

The data for this research will be collected from an organization
based in the Netherlands. We are looking for reported phishing emails
in 2013. We will be enumerating our methodology into several steps:

1. Raw data collection

2. Data categorization

3. Variable establishment

5. Data classification

6. Analysis

Data categorization will be determined by several aspects; languages
(Dutch, English, others) and whether the email is indeed a phish,
spam or legitimate. We predict there will be duplicated phishing emails
so that we will only consider only unique email in our dataset. 

We have studied the structural properties of phishing emails in the
previous section and we will establish our variable based on these
properties. Furthermore, the data classification phase will be indicated
by our coding of the categorization result into these variables. Lastly,
the data will be analyzed from mainly three different viewpoints;
properties characteristics, persuasion principles characteristics
and their relationships.


\part*{Conclusion}

In this report, we have tried to summarize all relevant literature.
We have outlined the basic understanding related to general phishing
in respect to its history. We have also reviewed its universal definition
from various sources. One study aggregated and systematically achieved
the consensual definition of phishing in which we follow. We have
also briefly explain the damage of phishing in term of direct cost.
We have learned that some organizations which measures the direct
cost of phishing attacks are being biased and thus we doubted the
state of it. However, we believe that phishing is still an issue in
terms of trust among users and the misuse of email as a means of communications.
We have learned that human factor is one of the most critical aspects
in phishing. Many studies suggest several phases when phishing attack
occurred and we tried to synthesized these phases and determined our
own phases based on the attacker point of view. We have outlined the
general type of phishing includes phishing based visual similarities
and malware based phishing. Many studies suggest both technical and
social countermeasures such as restriction list, machine learning
approach to detect spoofed website and anti phishing training. However,
we have found an absence of empirical research conducted to investigate
the role of persuasive elements as a human factor in respect to phishing
email which can be an aspect to be considered to prevent phishing
attack.

We have formulated two research questions and some hypotheses based
on our impressions from literature. We have also outlined a methodology
to perform the data analysis that hopefully can answer the research
questions from its findings. THIS IS TEST SEPARATING BIBLIOGRAPHY
\citep{moore2009economics}.


\part*{Bibliography}

\bibliographystyle{IEEEtran}
\begin{btSect}{Bibliography}
\btPrintCited
\end{btSect}



\section*{Aditional}

\bibliographystyle{IEEEtran}
\begin{btSect}{additional}
\btPrintCited
\end{btSect}

\end{document}
